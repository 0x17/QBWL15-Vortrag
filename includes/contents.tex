
%==========================================================================================
\section{Ressourcenbeschränkte Projektplanung und Zusatzkapazitäten}

\begin{frame}
\frametitle{Gliederung}
\tableofcontents[current, hideallsubsections]
\end{frame}

\begin{frame}[t]
\frametitle{Ressourcenbeschränkte Projektplanung}
\begin{center}
\includegraphics<1>[page=1, width=\textwidth]{images/rcpsp.pdf}
\includegraphics<2>[page=2, width=\textwidth]{images/rcpsp.pdf}
\includegraphics<3>[page=3, width=\textwidth]{images/rcpsp.pdf}\\
\end{center}

{\small
Projektdauerminimale Einplanung von Arbeitsgängen $j$ mit gegebenen
\begin{itemize}
\itemsep0em
\item<2-3> Dauern $d_j$ \textcolor{gray}{$=(0, 3, 2, 2, 3, 1, 2, 0)^T$}
\item<3> Ressourcenbelastungen $k_{jr}$ \textcolor{gray}{$=(0, 3, 2, 2, 1, 2, 1, 0)^T$}
\item<2-3> Reihenfolgebeziehungen $i \in \mathcal{P}_j $ \textcolor{gray}{$=(\emptyset, \{0\}, \{0\}, \ldots, \{4,5\}, \{6\})$}
\item<3> Kapazitätsrestriktionen $K_r$ \textcolor{gray}{$=(4)$}
\end{itemize}
}
\end{frame}

%==========================================================================================

\begin{frame}
\frametitle{Projektdauer und Deckungsbeitrag}
\begin{small}
Praxisbeispiel: Aufarbeitung eines Triebwerks durch Dienstleister
\begin{itemize}
\item<1-3> Projektdauer {\large $\downarrow$} $\implies$ Zahlungsbereitschaft {\large $\uparrow$\\}
\item<2-3> Überstunden {\large $\uparrow$} $\implies$ Projektdauer {\large $\downarrow$} Kosten {\large $\uparrow$}\\
\item<3>[] \begin{tabbing}
$\Rightarrow$ \= Maximierung des Deckungsbeitrags als Trade-off zwischen\\
\>Dauer- und Kostenminimierung
\end{tabbing}

\end{itemize}
\end{small}
\begin{center}
\includegraphics<1>[page=1,scale=0.29]{images/ErloesKostenDeckungsbeitrag.pdf}
\includegraphics<2>[page=2,scale=0.29]{images/ErloesKostenDeckungsbeitrag.pdf}
\includegraphics<3>[page=3,scale=0.29]{images/ErloesKostenDeckungsbeitrag.pdf}
\end{center}
\end{frame}

%%%%%%%%%%%%%%%%%%%%%%%%%%%%%%%%%%%%%%%%%%%%%%%%%%%%%%%%%%%%%%%%%%%%%%%%%%

\begin{frame}
\frametitle{Konzeptionelles Modell: RCPSP\only<2>{-OC}}
\begin{small}
\begin{itemize}
\item Zielfunktion \only<1>{\[\mbox{min } FT_J \textcolor{white}{- \sum_{r \in \mathcal{R}} \sum_{t \in \mathcal{T}} \kappa_r z_{rt}} \]\\[4.5mm]}\only<2>{\textcolor{red}{\[\mbox{max } u_{FT_J} - \sum_{r \in \mathcal{R}} \sum_{t \in \mathcal{T}} \kappa_r z_{rt}\]}}
\item Reihenfolgerestriktionen \[FT_i + d_i \leq FT_j \quad\quad\quad\quad j \in \mathcal{J}, i \in \mathcal{P}_j\]
\item Kapazitätsrestriktionen \[\sum_{j \in \mathcal{S}_t} k_{jr} \leq K_r\only<1>{\textcolor{white}{+ z_{rt}}}\only<2>{\textcolor{red}{+ z_{rt}}} \quad\quad\quad\; r \in \mathcal{R}, t \in \mathcal{T} \]
\[\mathcal{S}_t = \{j\;|\;(FT_j-d_j) < t \leq FT_j\}\]
\item<2> Obere Schranke für Zusatzkapazität \textcolor{red}{\[z_{rt} \leq \overline{z}_r \quad\quad\quad\quad\quad\quad\quad\quad r \in \mathcal{R}, t \in \mathcal{T}\]}
\end{itemize}
\end{small}
\end{frame}

%%%%%%%%%%%%%%%%%%%%%%%%%%%%%%%%%%%%%%%%%%%%%%%%%%%%%%%%%%%%%%%%%%%%%%%%%%%%%%%%%%

\begin{frame}[noframenumbering]
	\frametitle{Erzeugung relevanter Probleminstanzen}
	\begin{itemize}
		\item Obere Schranke für Zusatzkapazität $\overline{z}_r=\frac{1}{2}K_r$
		\item Kostensatz für Zusatzkapazität $\kappa_r=\frac{1}{2}\mbox{GE}$
		\item Projektdauerabhängiger Erlös \[u_{t}=C^{\mbox{max}} - \frac{C^{\mbox{max}}}{(T^{\mbox{max}}-T^{\mbox{min}})^2} \cdot (t-T^{\mbox{min}})^2\]
	\end{itemize}
	\begin{center}
		\includegraphics[scale=0.26]{images/DeadlineCosts.pdf}
	\end{center}
\end{frame}

%%%%%%%%%%%%%%%%%%%%%%%%%%%%%%%%%%%%%%%%%%%%%%%%%%%%%%%%%%%%%%%%%%%%%%%%%%%%%%%%%%

\begin{frame}[t]
\frametitle{Ressoucenbeschränkte Projektplanung\\mit Zusatzkapazitäten (RCPSP-OC)}
\includegraphics<1>[page=1, scale=0.58]{images/RCPSPOCDiagram.pdf}
\includegraphics<2>[page=2, scale=0.58]{images/RCPSPOCDiagram.pdf}
\includegraphics<3>[page=3, scale=0.58]{images/RCPSPOCDiagram.pdf}\\
\begin{center}
\only<1>{
\begin{tabbing}
Überstundenkosten: \= 0 GE\\
Projektdauer: \> 10 Perioden $\rightarrow$ Erlös: 0 GE\\
Deckungsbeitrag: \> 0 GE - 0 GE = 0 GE
\end{tabbing}
}
\only<2>{
\begin{tabbing}
Überstundenkosten: \= 1 GE\\
Projektdauer: \> 8 Perioden $\rightarrow$ Erlös: 1 GE\\
Deckungsbeitrag: \> 1 GE - 1 GE = 0 GE
\end{tabbing}
}
\only<3>{
\begin{tabbing}
Überstundenkosten: \= 0{,}5 GE\\
Projektdauer: \> 9 Perioden $\rightarrow$ Erlös: 0{,}75 GE\\
Deckungsbeitrag: \> 0{,}75 GE - 0{,}5 GE = 0{,}25 GE
\end{tabbing}
}
\end{center}
\end{frame}

%==========================================================================================

\section{Heuristische Lösungsverfahren}

\begin{frame}
\frametitle{Gliederung}
\tableofcontents[current,currentsubsection]
\end{frame}

\begin{frame}
\frametitle{Notwendigkeit heuristischer Lösung}
\begin{itemize}
\item RCPSP ist $\mathcal{NP}$-schweres Problem
\item und RCPSP $\preceq_p$ RCPSP-OC: $\overline{z}_{r}=0, u_t=-t$
\item[] $\implies$ RCPSP-OC ist $\mathcal{NP}$-schweres Problem\\[10mm]
\item Exakte Lösungsverfahren für praxisnahe Problemgrößen nicht handhabbar
\item[] $\rightarrow$ Heuristik
\end{itemize}
\end{frame}

\begin{frame}
\frametitle{Motivation zur Verwendung eines seriellen Schedule Generation Scheme (SSGS)}
\begin{itemize}
\item Dominierende Heuristiken in Untersuchung von RCPSP-Heuristiken durch {\footnotesize \cite{Kolisch2006}}:\\[6mm] Metaheuristiken basierend auf
\begin{itemize}
\item Seriellem Schedule Generation Scheme (SSGS)
\item Aktivitätenlistenrepräsentation, zum Beispiel:
\end{itemize}
\end{itemize}
\[\lambda=(0,1,2,4,3,5,6,7)\]
\end{frame}

\begin{frame}[t]
\frametitle{Beispiel: Ablauf des SSGS}
\begin{center}
\includegraphics<1>[page=1, scale=0.68]{images/ssgs.pdf}
\includegraphics<2>[page=2, scale=0.68]{images/ssgs.pdf}
\includegraphics<3>[page=3, scale=0.68]{images/ssgs.pdf}
\includegraphics<4>[page=4, scale=0.68]{images/ssgs.pdf}
%\includegraphics<5>[page=5, scale=0.68]{images/ssgs.pdf}
%\includegraphics<6>[page=6, scale=0.68]{images/ssgs.pdf}
\includegraphics<5>[page=7, scale=0.68]{images/ssgs.pdf}\\
$\lambda=(0,\textbf<1>{1},\textbf<2>{2},\textbf<3>{4},\textbf<4>{3},5,6,7)$
\only<5>{
\\[1mm]
\begin{small}
\textbf{Problem 1:} SSGS nur bei gegebenen Kapazitäten anwendbar
\end{small}
}
\end{center}
\end{frame}

\begin{frame}
\frametitle{Bestimmung Aktivitätenliste $\lambda$}
\begin{itemize}
\item[] \textbf{Problem 2:}\\Lösungsgüte von Aktivitätenliste $\lambda$ abhängig\\[7mm]
\item Vollständige Enumeration topologischer Sortierungen nicht handhabbar\\
$\rightarrow$ Suchraum mit Metaheuristik erkunden\\[6mm]
\item Erster Kandidat: Genetischer Algorithmus
\begin{itemize}
\item {\footnotesize 3 der Top 5 RCPSP-Heuristiken nach \cite{Kolisch2006}}
\item {\footnotesize Kann Problem 1 und 2 lösen}
\end{itemize}
\end{itemize}
\end{frame}

\begin{frame}
\frametitle{Genetische Algorithmen}
Repräsentationen im Überblick
\begin{itemize}
	\item \small{Startzeit zwischen Reihenfolge- und Ressourcenzulässigkeit}\\[2mm]
	\begin{small}
	\begin{tabular}{cp{7.5cm}}
	\hline
	Individuum & Planerzeugung\\
	\hline
	$(\lambda)$ & Vervollständigung jeder Alternative ohne ZK\\	
	$(\lambda|\tau)$& Startzeit beliebig in Zeitfenster $\tau \in [0,1)$\\
	$(\lambda|\beta)$& Startzeit an Zeitfensterrand $\beta \in \{0,1\}$\\
	\end{tabular}
	\end{small}\\[4mm]
	
	\item \small{Vorbestimmung des Ressourcenprofils}\\[2mm]
	\begin{small}
		\begin{tabular}{cp{7.5cm}}
			\hline
			Individuum & Planerzeugung\\
			\hline
			$(\lambda|\tilde{z}_{rt})$ & Kapazität entspricht $K_r+z_{rt}$ mit $z_{rt} \leq \tilde{z}_{rt}$ \\
			$(\lambda|\tilde{z}_r)$ & Kapazität entspricht $K_r+z_r$ mit $z_{rt} \leq \tilde{z}_{r}$\\
		\end{tabular}
	\end{small}
\end{itemize}
\end{frame}

%%%%%%%%%%%%%%%%%%%%%%%%%%%%%%%%%%%%%%%%%%%%%%%%%%%%%%%%%%%%%%%%%%%%%%%%%%%%%%%%%%%%%%%%%%%%%%%%%%%%%%%

\subsection{Teilplanvervollständigung $(\lambda)$}
\begin{frame}
	\frametitle{Gliederung}
	\tableofcontents[currentsubsection]
\end{frame}

\begin{frame}[t]
	\frametitle{Planerzeugung $(\lambda)$: Einplanung AG 2}
	\begin{center}
		\includegraphics<1>[page=1, scale=0.65]{images/ssgsoc.pdf}
		\includegraphics<2>[page=2, scale=0.65]{images/ssgsoc.pdf}
		\includegraphics<3>[page=3, scale=0.65]{images/ssgsoc.pdf}
		\includegraphics<4>[page=4, scale=0.65]{images/ssgsoc.pdf}
		\includegraphics<5>[page=5, scale=0.65]{images/ssgsoc.pdf}
		\includegraphics<6>[page=6, scale=0.65]{images/ssgsoc.pdf}
		\includegraphics<7>[page=5, scale=0.65]{images/ssgsoc.pdf}\\
		
		\only<1>{
			Bisher Arbeitsgänge 0, 1, 2 und 3 eingeplant.
			\vspace*{12.5mm}
		}
		
		\only<2>{
			Früheste Reihenfolgezulässigkeit $\underline{t}$\\
			AG 2 mit $ST_2=\underline{t}=1$ einplanen.
			%\vspace*{15mm}
			\textcolor{white}{$\mbox{DB} = u_{8}-\sum_r \sum_t \kappa_r\mbox{z}_{rt} = 1\mbox{ GE} - 4\mbox{ GE} = 3\mbox{ GE}$}
		}
		
		\only<3>{
			Früheste Reihenfolgezulässigkeit $\underline{t}$\\
			AG 2 mit $ST_2=\underline{t}=1$ einplanen.\\
			$\mbox{DB} = u_{8}-\sum_r \sum_t \kappa_r z_{rt} = 1\mbox{ GE} - 1\mbox{ GE} = 0\mbox{ GE}$
		}
		
		\only<4>{
			Inkrementiere\\
			AG 2 mit $ST_2=2$ einplanen.\\
			$\mbox{DB} = u_{8}-\sum_r \sum_t \kappa_r z_{rt} = 1\mbox{ GE} - 1\mbox{ GE} = 0\mbox{ GE}$
		}
		
		\only<5>{
			Inkrementiere\\
			AG 2 mit $ST_2=3$ einplanen.\\
			$\mbox{DB} = u_{9}-\sum_r \sum_t \kappa_r z_{rt} = \frac{3}{4}\mbox{ GE} - \frac{1}{2}\mbox{ GE} = \frac{1}{4}\mbox{ GE}$
		}
		
		\only<6>{
			Früheste Ressourcenzulässigkeit $\overline{t}$\\
			AG 2 mit $ST_4=\overline{t}=4$ einplanen.\\
			$\mbox{DB} = u_{10}-\sum_r \sum_t \kappa_r z_{rt} = 0\mbox{ GE} - 0\mbox{ GE} = 0\mbox{ GE}$
		}
		
		\only<7>{
			Bester Kandidat für Einplanungszeitpunkt\\
			AG 2 mit $ST_4=3$ einplanen.\\
			$\mbox{DB} = u_{10}-\sum_r \sum_t \kappa_r z_{rt} = \frac{3}{4}\mbox{ GE} - \frac{1}{2}\mbox{ GE} = \frac{1}{4}\mbox{ GE}$
		}
	\end{center}
\end{frame}

\begin{frame}
	\frametitle{Genetischer Algorithmus $(\lambda)$}
	\begin{small}
		\begin{center}
			\begin{tabular}{rl}
				\hline 
				Individuum & $(\lambda)=(0,1,4,2,5,3,6,7)$\parbox[c][40pt][c]{0pt}{}\tabularnewline
				\hline 
				Initialpopulation & Regret-Based Biased Random Sampling (LFTs)\tabularnewline
				\hline 
				Rekombination & One Point Crossover\tabularnewline
				\hline 
				Mutation & Neighborhood Swap\tabularnewline
				\hline 
				Fitness & DB von Plan erzeugt durch SSGS-OC\tabularnewline
				\hline 
			\end{tabular}
		\end{center}
	\end{small}
\end{frame}

%%%%%%%%%%%%%%%%%%%%%%%%%%%%%%%%%%%%%%%%%%%%%%%%%%%%%%%%%%%%%%%%%%%%%%%%%%%%%%%%%%%%%%%%%%%%%%%%%%%%%%%

\subsection{Wahl in Zeitfenster durch GA $(\lambda|\tau)$}
\begin{frame}
	\frametitle{Gliederung}
	\tableofcontents[currentsubsection]
\end{frame}

\begin{frame}[t]
	\frametitle{Planerzeugung $(\lambda|\tau)$: Einplanung von 2}
	\includegraphics<1-2>[page=1, scale=0.7]{images/ssgstau.pdf}
	\includegraphics<3>[page=2, scale=0.7]{images/ssgstau.pdf}
	\only<1>{\[ ST_2 = \overline{t} - [ (\overline{t}-\underline{t}) \cdot \tau ] \]}
	\only<2>{\[ ST_2 = 4 - [ (4-1) \cdot 0{,}3 ] = 4 - [ 0{,}9 ] = 3\]}
	\only<3>{\[ ST_2 = 4 - [ (4-1) \cdot 0{,}9 ] = 4 - [ 2{,}7 ] = 1\]}
\end{frame}

\begin{frame}
	\frametitle{Genetischer Algorithmus $(\lambda|\tau)$}
	\begin{small}
		\begin{center}
			\begin{tabular}{rl}
				\hline 
				Individuum & $\begin{pmatrix}\lambda\\\tau\end{pmatrix}=\begin{pmatrix}0,1,3,5,2,4,6,7\\0,0.3,0.5,0.8,0.9,0.2,0.1,0.2\end{pmatrix}$\parbox[c][40pt][c]{0pt}{}\tabularnewline
				\hline 
				Initialpopulation & $\tau_j=\mbox{rand} \in [0, 1) \; \forall j$\tabularnewline
				\hline 
				Rekombination & Gemeinsamer OPC mit $\lambda$\tabularnewline
				\hline 
				Mutation & $\tau_j=\mbox{rand } \in [0,1)$ mit $P_{mutate}$\tabularnewline
				\hline 
				Fitness & DB von Plan erzeugt durch SSGS\tabularnewline
				\hline 
			\end{tabular}
		\end{center}
	\end{small}
\end{frame}

%%%%%%%%%%%%%%%%%%%%%%%%%%%%%%%%%%%%%%%%%%%%%%%%%%%%%%%%%%%%%%%%%%%%%%%%%%%%%%%%%%%%%%%%%%%%%%%%%%%%%%%

\subsection{Zeitfenstergrenzen $(\lambda|\beta)$}
\begin{frame}
	\frametitle{Gliederung}
	\tableofcontents[currentsubsection]
\end{frame}

\begin{frame}
	\frametitle{Planerzeugung $(\lambda|\beta)$: untere Einplanung}
	\includegraphics<1>[page=1, scale=0.75]{images/SSGSbetaLower.pdf}
	\includegraphics<2>[page=2, scale=0.75]{images/SSGSbetaLower.pdf}
\end{frame}

\begin{frame}
	\frametitle{Planerzeugung $(\lambda|\beta)$: obere Einplanung}
	\includegraphics<1>[page=1, scale=0.75]{images/SSGSbetaUpper.pdf}
	\includegraphics<2>[page=2, scale=0.75]{images/SSGSbetaUpper.pdf}
	\includegraphics<3>[page=3, scale=0.75]{images/SSGSbetaUpper.pdf}
	\includegraphics<4>[page=4, scale=0.75]{images/SSGSbetaUpper.pdf}
\end{frame}

\begin{frame}
	\frametitle{$(\lambda|\beta)$-Varianten}
	\begin{itemize}
		\item Arbeitsgangzuordnung: AG $j=\lambda_i$ nutzt ZK, gdw.
		\begin{itemize}
			\item $\beta_i=1$ (Listenposition assoziiert)
			\item $\beta_j=1$ (AG-Nummer assoziiert)\\[4mm]
		\end{itemize}
		\item Einplanungsart
		\begin{itemize}
			\item "`von unten"': Nutze erst Normalkapazität
			\item "`von oben"': Nutze erst Überstunden\\[4mm]
		\end{itemize}
		\item Crossover
		\begin{itemize}
			\item Gemeinsamer One Point Crossover
			\item Zwei getrennte One Point Crossovers\\[5mm]
		\end{itemize}
		\item[$\implies$] Insgesamt $2 \cdot 2 \cdot 2 = 8$ unterschiedliche Varianten
	\end{itemize}
\end{frame}

\begin{frame}
	\frametitle{Genetischer Algorithmus $(\lambda|\beta)$}
	\begin{small}
		\begin{center}
			\begin{tabular}{rl}
				\hline 
				Individuum & $\begin{pmatrix}\lambda\\\beta\end{pmatrix}=\begin{pmatrix}0,1,3,5,2,4,6,7\\0,1,1,0,1,0,1,0\end{pmatrix}$\parbox[c][40pt][c]{0pt}{}\tabularnewline
				\hline 
				Initialpopulation & $\beta_i=\mbox{rand} \in \{0,1\} \; \forall i \in \mathcal{J}$\tabularnewline
				\hline 
				Rekombination & One Point Crossover\tabularnewline
				\hline 
				Mutation & $\beta_i=\neg \beta_i$ mit $P_{mutate}$\tabularnewline
				\hline 
				Fitness & DB von Plan erzeugt durch SSGS\tabularnewline
				\hline 
			\end{tabular}
		\end{center}
	\end{small}
\end{frame}

%%%%%%%%%%%%%%%%%%%%%%%%%%%%%%%%%%%%%%%%%%%%%%%%%%%%%%%%%%%%%%%%%%%%%%%%%%%%%%%%%%%%%%%%%%%%%%%%%%%%%%%

\subsection{Ressourcenprofil durch GA $(\lambda|z_{rt})$}
\begin{frame}
	\frametitle{Gliederung}
	\tableofcontents[currentsubsection]
\end{frame}

\begin{frame}
	\frametitle{Planerzeugung $(\lambda|z_{rt})$: Einplanung von 2}
	\includegraphics<1>[page=1, scale=0.75]{images/SSGSzrt.pdf}
	\includegraphics<2>[page=2, scale=0.75]{images/SSGSzrt.pdf}
\end{frame}

\begin{frame}
	\frametitle{Genetischer Algorithmus $(\lambda|z_{rt})$}
	\begin{small}
		\begin{center}
			\begin{tabular}{rl}
				\hline 
				Individuum & $(\lambda|\tilde{z}_{rt})=(0,1,3,5,2,4,6,7|\begin{pmatrix} 0 & 2 & \ldots\\ \vdots & \ddots \end{pmatrix})$\parbox[c][40pt][c]{0pt}{}\tabularnewline
				\hline 
				Initialpopulation & $\tilde{z}_{rt}=\mbox{rand} \in \{0,\ldots,\overline{z}_{r}\}\;\forall r,t$\tabularnewline
				\hline 
				Rekombination & One Point Crossover\tabularnewline
				\hline 
				Mutation & $\tilde{z}_{rt}=\mbox{rand} \in \{0, \ldots, \overline{z}_{r}\}$ mit $P_{mutate}$\tabularnewline
				\hline 
				Fitness & DB von Plan erzeugt durch SSGS\tabularnewline
				\hline 
			\end{tabular}
		\end{center}
	\end{small}
\end{frame}

%%%%%%%%%%%%%%%%%%%%%%%%%%%%%%%%%%%%%%%%%%%%%%%%%%%%%%%%%%%%%%%%%%%%%%%%%%%%%%%%%%%%%%%%%%%%%%%%%%%%%%%

\subsection{Feste Kapazität durch GA $(\lambda|z_{r})$}
\begin{frame}
	\frametitle{Gliederung}
	\tableofcontents[currentsubsection]
\end{frame}


\begin{frame}
	\frametitle{Planerzeugung $(\lambda|z_{r})$: Einplanung von 2}
	\includegraphics<1>[page=1, scale=0.75]{images/SSGSzr.pdf}
	\includegraphics<2>[page=2, scale=0.75]{images/SSGSzr.pdf}
\end{frame}

\begin{frame}
	\frametitle{Genetischer Algorithmus $(\lambda|z_{r})$}
	\begin{small}
		\begin{center}
			\begin{tabular}{rl}
				\hline 
				Individuum & $(\lambda|\tilde{z}_{r})=(0,1,3,5,2,4,6,7|0,2,2,0,3,0,\ldots)$\parbox[c][40pt][c]{0pt}{}\tabularnewline
				\hline 
				Initialpopulation & $\tilde{z}_{r}=\mbox{rand} \in \{0, \ldots, \overline{z}_{r}\} \; \forall r$\tabularnewline
				\hline 
				Rekombination & One Point Crossover\tabularnewline
				\hline 
				Mutation & $\tilde{z}_{r}=\mbox{rand} \in \{0, \ldots, \overline{z}_{r}$\} mit $P_{mutate}$\tabularnewline
				\hline 
				Fitness & DB von Plan erzeugt durch SSGS\tabularnewline
				\hline
			\end{tabular}
		\end{center}
	\end{small}
\end{frame}

%%%%%%%%%%%%%%%%%%%%%%%%%%%%%%%%%%%%%%%%%%%%%%%%%%%%%%%%%%%%%%%%%%%%%%%%%%%%%%%%%%%%%%%%%%%%%%%%%%%%%%%

\section{Numerische Ergebnisse}
\begin{frame}
\frametitle{Gliederung}
\tableofcontents[current, hidesubsections]
\end{frame}


\begin{frame}[t]
\frametitle{Numerische Ergebnisse}
\begin{footnotesize}
\textbf{Parameter:} Populationsgröße$=80$, $P_{mutate}=5\%$\\
\textbf{Testinstanzen:} PSPLIB\\
j30: 199 optimal in $<$2 Stunden lösbare kapazitätsbeschränkte Projekte, Zeitlimit $=1s$\\
j120: 600 Projekte, Zeitlimit $=15s$\\[3mm]
Deckungsbeiträge $\geq 0$ aufgrund $T^{\mbox{max}}$-Wahl in $u_t$ $\implies$ Rationalskaliert

\begin{center}	
\tabcolsep=0.16cm
\begin{tabular}{|c|rrrc|rrrc|}
	\hline
	& \multicolumn{4}{c|}{j30} & \multicolumn{4}{c|}{j120}\\
	 & $\varnothing$ Gap & $\overline{\mbox{Gap}}$ & Opt & $\varnothing$ Rang & $\varnothing$ Gap & $\overline{\mbox{Gap}}$ & BBL & $\varnothing$ Rang \\[3pt]
	\hline
   $(\lambda|z_{rt})$&1.41\%&8.83\%&46.73\%&1.98&1.91\%&12.38\%&21.20\%&3.63\\
	\hline
   $(\lambda|z_r)$&3.01\%&45.00\%&28.64\%&2.60&0.95\%&8.30\%&52.14\%&2.19\\
	\hline
	$(\lambda|\beta)$&2.80\%&45.00\%&29.65\%&2.37&1.61\%&8.31\%&25.98\%&2.97\\
	best & \multicolumn{4}{c|}{AG, Gem. OPC, oberes SGS} & \multicolumn{4}{c|}{Liste, Gem. OPC, unteres SGS}\\
	\hline
	$(\lambda|\beta)$&5.27\%&45.00\%&21.11\%&4.39&11.76\%&28.25\%&8.55\%&9.56\\
	worst & \multicolumn{4}{c|}{Liste, Getr. OPCs, oberes SGS} & \multicolumn{4}{c|}{Liste, Getr. OPCs, oberes SGS}\\
	\hline
	$(\lambda|\tau)$&2.45\%&20.23\%&41.21\%&2.98&4.48\%&14.17\%&2.56\%&6.08\\
	\hline
	$(\lambda)$&1.20\%&14.29\%&52.76\%&1.78&5.04\%&19.18\%&27.52\%&5.40\\
	\hline
\end{tabular}
\end{center}

\end{footnotesize}	

\end{frame}

\section{Ausblick}

\begin{frame}
\frametitle{Gliederung}
\tableofcontents[current, hidesubsections]
\end{frame}

\begin{frame}
\frametitle{Ausblick}
\begin{itemize}
\item Heuristische Methoden
	\begin{itemize}
	\item Techniken führender RCPSP-Heuristiken
		\begin{itemize}
		\item Vorwärts-Rückwärts-Verbesserung
		\item Peak-Crossover
		\item Zweite Suchphase in Nachbarschaft
		\end{itemize}
	\item Stellschrauben im GA: $N^G$, $N^I$, $P_{mutate}$\\[4mm]
	\end{itemize}
	
\item Exakte Methoden
	\begin{itemize}\item Problemspezifischer Branch\&Bound-Algorithmus\\[4mm]\end{itemize}
	
\item Problemgeneralisierung
	\begin{itemize}
	\item Zeitabhängige Ressourcenbelastungen $k_{jr\tau}$
	\item Mehr-Modus-Fall
	\item Flexible Projekte
	\end{itemize}
\end{itemize}
\end{frame}




