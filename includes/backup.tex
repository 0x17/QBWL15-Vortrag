\begin{frame}[noframenumbering]
\vfill
\begin{center}
\begin{beamercolorbox}[sep=8pt,center]{title}
\usebeamerfont{title}
Backup-Folien
\end{beamercolorbox}
\end{center}
\vfill
\end{frame}

%%%%%%%%%%%%%%%%%%%%%%%%%%%%%%%%%%%%%%%%%%%%%%%%%%%%%%%%%%%%%%%%%%%%%%%%%%

\begin{frame}[noframenumbering]
	\frametitle{Konzeptionelles Modell: RCPSP\only<2>{-OC}}
	\begin{small}
		\begin{itemize}
			\item Zielfunktion \only<1>{\[\mbox{min } FT_J \textcolor{white}{- \sum_{r \in \mathcal{R}} \sum_{t \in \mathcal{T}} \kappa_r z_{rt}} \]\\[4.5mm]}\only<2>{\textcolor{red}{\[\mbox{max } u_{FT_J} - \sum_{r \in \mathcal{R}} \sum_{t \in \mathcal{T}} \kappa_r z_{rt}\]}}
			\item Reihenfolgerestriktionen \[FT_i + d_i \leq FT_j \quad\quad\quad\quad j \in \mathcal{J}, i \in \mathcal{P}_j\]
			\item Kapazitätsrestriktionen \[\sum_{j \in \mathcal{S}_t} k_{jr} \leq K_r\only<1>{\textcolor{white}{+ z_{rt}}}\only<2>{\textcolor{red}{+ z_{rt}}} \quad\quad\quad\; r \in \mathcal{R}, t \in \mathcal{T} \]
			\[\mathcal{S}_t = \{j\;|\;(FT_j-d_j) < t \leq FT_j\}\]
			\item<2> Obere Schranke für Zusatzkapazität \textcolor{red}{\[z_{rt} \leq \overline{z}_r \quad\quad\quad\quad\quad\quad\quad\quad r \in \mathcal{R}, t \in \mathcal{T}\]}
		\end{itemize}
	\end{small}
\end{frame}

%%%%%%%%%%%%%%%%%%%%%%%%%%%%%%%%%%%%%%%%%%%%%%%%%%%%%%%%%%%%%%%%%%%%%%%%%%%%%%%%%%

\begin{frame}[noframenumbering]
	\frametitle{Entscheidungsmodell: RCPSP\only<2>{-OC}}
	\begin{footnotesize}
		\begin{itemize}
			\item Zielfunktion\\[-7mm]
			\[
			\only<1>{\mbox{min } \sum_{t=EFT_{J+1}}^{LFT_{J+1}} t \cdot x_{{J+1},t} \textcolor{white}{\enspace - \sum_{r \in \mathcal{R}} \sum_{t \in \mathcal{T}} \kappa_r \cdot z_{rt}}}\only<2>{\textcolor{red}{\mbox{max } \sum_{t=EFT_{J+1}}^{LFT_{J+1}} u_t \cdot x_{{J+1},t} - \sum_{r \in \mathcal{R}} \sum_{t \in \mathcal{T}} \kappa_r \cdot z_{rt}}}
			\]
			
			\item Einmalige Durchführung
			\[
			\sum_{t=EFT_j}^{LFT_j} x_{jt} = 1 \,\,\,\quad\quad\quad\quad\quad\quad\quad\quad\quad\quad\quad j \in \mathcal{J}\textcolor{white}{, t \in \mathcal{T}}
			\]
			
			\item Reihenfolgerestriktionen
			\[
			\sum_{t=EFT_i}^{LFT_i} x_{it} \cdot t \leq \sum_{t=EFT_j}^{LFT_j} x_{jt} \cdot t - d_j \,\,\:\:\:\quad\quad\quad j \in \mathcal{J}, \; i \in \mathcal{P}_j
			\]
			
			\item Kapazitätsrestriktionen
			\[
			\sum_{j=1}^{J} \sum_{\tau=t}^{t+d_j-1} k_{jr} \cdot x_{j\tau} \leq K_r\only<1>{\textcolor{white}{ + z_{rt}}}\only<2>{ \textcolor{red}{+ z_{rt}}} \,\,\:\:\:\:\:\quad\quad\quad\quad r \in \mathcal{R}, \; t \in \mathcal{T}
			\]
			
			\item<2> Obere Schranke für Zusatzkapazität\\[-3mm]
			\[
			\textcolor{red}{z_{rt} \leq \overline{z}_r} \,\;\;\quad\quad\quad\quad\quad\quad\quad\quad\quad\quad\quad\quad\quad\quad r \in \mathcal{R}, \; t \in \mathcal{T}
			\]
		\end{itemize}
	\end{footnotesize}
\end{frame}

%%%%%%%%%%%%%%%%%%%%%%%%%%%%%%%%%%%%%%%%%%%%%%%%%%%%%%%%%%%%%%%%%%%%%%%%%%%%%%%%%%%%%%%%%%

\begin{frame}[t, noframenumbering]
	\frametitle{Ressoucenbeschränkte Projektplanung\\mit Zusatzkapazitäten (RCPSP-OC)}
	\begin{center}
		\includegraphics<1>[page=1, scale=0.75]{images/RCPSPOCDiagram.pdf}
		\includegraphics<2>[page=2, scale=0.75]{images/RCPSPOCDiagram.pdf}
		\includegraphics<3>[page=3, scale=0.75]{images/RCPSPOCDiagram.pdf}
	\end{center}
\end{frame}

%%%%%%%%%%%%%%%%%%%%%%%%%%%%%%%%%%%%%%%%%%%%%%%%%%%%%%%%%%%%%%%%%%%%%%%%%%%%%%%%%%%%%%%%%%%%%%%%%%%%%%%%%%%%%%%%%%%%%%%%%%%%%%%%%%%%

\begin{frame}[noframenumbering]
\frametitle{Verwandte Probleme aus der Literatur}
\begin{itemize}
\item Variable Kapazitäten
\begin{itemize}
\item Ressourcenprofil als Parameter {\footnotesize \cite{Klein2000}, \cite{Hartmann2012}}
\item Ressourceninvestitionsproblem {\footnotesize \cite{Mohring1984}}
\item Ressourcenabweichungsproblem {\footnotesize \cite{Neumann2003}} 
\item Ressourcenüberladungsproblem {\footnotesize \cite{Neumann2003}}
\end{itemize}
\vspace*{4mm}
\item Variable Ressourcennachfrage
\begin{itemize}
\item Flexible Ressourcenprofile {\footnotesize \cite{Ranjbar2010}}
\item Zeit-Kosten-Tradeoff-Problem {\footnotesize \cite{Demeulemeester1996}}
\end{itemize}
\end{itemize}

\end{frame}


%%%%%%%%%%%%%%%%%%%%%%%%%%%%%%%%%%%%%%%%%%%%%%%%%%%%%%%%%%%%%%%%%%%%%%%%%%%%%%%%%%%%%%%%%%%%%%%%%%%

\begin{frame}[noframenumbering]
	\frametitle{Notwendigkeit heuristischer Lösung}
	\begin{itemize}
		\item RCPSP ist $\mathcal{NP}$-schweres Problem
		\item und RCPSP $\preceq_p$ RCPSP-OC: $\overline{z}_{r}=0, u_t=-t$
		\item[] $\implies$ RCPSP-OC ist $\mathcal{NP}$-schweres Problem\\[10mm]
		\item Exakte Lösungsverfahren für praxisnahe Problemgrößen nicht handhabbar
		\item[] $\implies$ Heuristik
	\end{itemize}
\end{frame}

\begin{frame}[noframenumbering]
	\frametitle{Motivation zur Verwendung eines seriellen Schedule Generation Scheme (SSGS)}
	\begin{itemize}
		\item Dominierende Heuristiken in Untersuchung von RCPSP-Heuristiken durch {\footnotesize \cite{Kolisch2006}}:\\[6mm] Metaheuristiken basierend auf
		\begin{itemize}
			\item Seriellem Schedule Generation Scheme (SSGS)
			\item Aktivitätenlistenrepräsentation, zum Beispiel:
		\end{itemize}
	\end{itemize}
	\[\lambda=(0,1,2,4,3,5,6,7)\]
\end{frame}

\begin{frame}[t,noframenumbering]
	\frametitle{Ablauf des SSGS}
	\begin{center}
		\includegraphics<1>[page=1, scale=0.68]{images/ShortSSGS.pdf}
		\includegraphics<2>[page=2, scale=0.68]{images/ShortSSGS.pdf}
		\includegraphics<3>[page=3, scale=0.68]{images/ShortSSGS.pdf}
		\includegraphics<4>[page=4, scale=0.68]{images/ShortSSGS.pdf}
		\includegraphics<5>[page=5, scale=0.68]{images/ShortSSGS.pdf}\\
		$\lambda=(0,2,4,\textbf<1-4>{1},3,5,6,7)$
		\only<5>{
			\\[1mm]
			\begin{small}
				\textbf{Problem 1:} SSGS nur bei gegebenen Kapazitäten anwendbar
			\end{small}
		}
	\end{center}
\end{frame}

\begin{frame}[noframenumbering]
	\frametitle{Bestimmung Aktivitätenliste $\lambda$}
	\begin{itemize}
		\item[] \textbf{Problem 2:}\\Lösungsgüte von Aktivitätenliste $\lambda$ abhängig\\[6mm]
		
		\item Es gibt für jedes Projekt $\lambda$ für kürzesten Plan
		\item Vollständige Enumeration topologischer Sortierungen nicht handhabbar\\
		$\rightarrow$ Suchraum mit Metaheuristik erkunden\\[6mm]
		\item Erster Kandidat: Genetischer Algorithmus
		\begin{itemize}
			\item {\footnotesize 3 der Top 5 RCPSP-Heuristiken nach \cite{Kolisch2006}}
			\item {\footnotesize Kann Problem 1 und 2 lösen}
		\end{itemize}
	\end{itemize}
\end{frame}

%%%%%%%%%%%%%%%%%%%%%%%%%%%%%%%%%%%%%%%%%%%%%%%%%%%%%%%%%%%%%%%%%%%%%%%%%%%%%%%%%%%%%%%%%%%%%%%

\begin{frame}[noframenumbering]
\frametitle{Gemeinsamkeiten der GAs (\only<1>{1}\only<2>{2}/2)}

\only<1>{
\begin{itemize}
\item Aktivitätenliste $\lambda$ als Bestandteil der Repräsentation
\item Initialpopulation

\begin{itemize}\item Regret-Based Biased Random Sampling\end{itemize}
\item Kreuzung
	\begin{itemize}
	\item Paarbildung:\\Ziehe erstes Elternteil aus 50\% Besten und zweites aus restlichen 50\% noch nicht gewählter Individuen\\[4mm]
	\item Tochter (Sohn) per One Point Crossover, d.h.:
	\item[] Übernehme bis zufälliger Stelle $q$ von der Mutter (Vater), Rest in Reihenfolge vom Vater (Mutter)
	\end{itemize}
\end{itemize}
}
\only<2>{
\begin{itemize}
\item Mutation
\begin{itemize}\item Mit Wahrscheinlichkeit $P_{mutate}$ vertausche AG mit Nachbarn, falls zulässig (Swap Neighborhood)\\[3mm]\end{itemize}
\item Selektion
\begin{itemize}\item Sortiere Gesamtpopulation (Eltern \& Kinder) nach absteigender Fitness. Neue Elterngeneration ist ``vordere'' Hälfte (Ranking Method)\\[3mm]\end{itemize}
\item Fitness \begin{itemize}\item DB von durch Individuum per SSGS induziertem Plan\end{itemize}
\end{itemize}
}
\end{frame}

%%%%%%%%%%%%%%%%%%%%%%%%%%%%%%%%%%%%%%%%%%%%%%%%%%%%%%%%%%%%%%%%%%%%%%%%%%%%%%%%%%%%%%%%%%%%%%%%%%%%%%%

%\subsection{Ressourcenprofil $(\lambda|z_{rt})$}
%\begin{frame}[noframenumbering]
%	\frametitle{Gliederung}
%	\tableofcontents[currentsubsection]
%\end{frame}

\begin{frame}[noframenumbering]
	\frametitle{Planerzeugung $(\lambda|z_{rt})$: Einplanung AG 2}
	\includegraphics<1>[page=1, scale=0.75]{images/SSGSzrt.pdf}
	\includegraphics<2>[page=2, scale=0.75]{images/SSGSzrt.pdf}
\end{frame}

\begin{frame}[noframenumbering]
	\frametitle{Genetischer Algorithmus $(\lambda|z_{rt})$}
	\begin{small}
		\begin{center}
			\begin{tabular}{rl}
				\hline 
				Individuum & $(\lambda|z_{rt})=(0,1,3,5,2,4,6,7|\begin{pmatrix} 0 & 2 & \ldots\\ \vdots & \ddots \end{pmatrix})$\parbox[c][40pt][c]{0pt}{}\tabularnewline
				\hline 
				Initialpopulation & $z_{rt}=$ Zufallszahl aus $\{0,\ldots,\overline{z}_{r}\}\;\forall r,t$\tabularnewline
				\hline 
				Rekombination & One Point Crossover\tabularnewline
				\hline 
				Mutation & $z_{rt}=$ Zufallszahl aus $\{0, \ldots, \overline{z}_{r}\}$ mit $P_{mutate}$\tabularnewline
				\hline 
				Fitness & Deckungsbeitrag von kodiertem Plan\tabularnewline
				\hline 
			\end{tabular}
		\end{center}
	\end{small}
\end{frame}

%%%%%%%%%%%%%%%%%%%%%%%%%%%%%%%%%%%%%%%%%%%%%%%%%%%%%%%%%%%%%%%%%%%%%%%%%%%%%%%%%%%%%%%%%%%%%%%%%%%%%%%

%\subsection{Zeitfenstergrenzen $(\lambda|\beta)$}
%\begin{frame}[noframenumbering]
%	\frametitle{Gliederung}
%	\tableofcontents[currentsubsection]
%\end{frame}

\begin{frame}[noframenumbering]
	\frametitle{Planerzeugung $(\lambda|\beta)$: \only<1-2>{untere}\only<3-6>{obere} Einplanung}
	
	\includegraphics<1>[page=1, scale=0.75]{images/SSGSbetaLower.pdf}
	\includegraphics<2>[page=2, scale=0.75]{images/SSGSbetaLower.pdf}
	
	\includegraphics<3>[page=1, scale=0.75]{images/SSGSbetaUpper.pdf}
	\includegraphics<4>[page=2, scale=0.75]{images/SSGSbetaUpper.pdf}
	\includegraphics<5>[page=3, scale=0.75]{images/SSGSbetaUpper.pdf}
	\includegraphics<6>[page=4, scale=0.75]{images/SSGSbetaUpper.pdf}
\end{frame}

\begin{frame}[noframenumbering]
	\frametitle{$(\lambda|\beta)$-Varianten}
	\begin{itemize}
		\item Arbeitsgangzuordnung: AG $j=\lambda_i$ nutzt ZK, gdw.
		\begin{itemize}
			\item $\beta_i=1$ (Listenposition assoziiert)
			\item $\beta_j=1$ (AG-Nummer assoziiert)\\[4mm]
		\end{itemize}
		\item Einplanungsart
		\begin{itemize}
			\item "`von unten"': Nutze erst Normalkapazität
			\item "`von oben"': Nutze erst Überstunden\\[4mm]
		\end{itemize}
		\item Crossover
		\begin{itemize}
			\item Gemeinsamer One Point Crossover
			\item Zwei getrennte One Point Crossovers\\[5mm]
		\end{itemize}
		\item[$\implies$] Insgesamt $2 \cdot 2 \cdot 2 = 8$ unterschiedliche Varianten
	\end{itemize}
\end{frame}

\begin{frame}[noframenumbering]
	\frametitle{Genetischer Algorithmus $(\lambda|\beta)$}
	\begin{small}
		\begin{center}
			\begin{tabular}{rl}
				\hline 
				Individuum & $\begin{pmatrix}\lambda\\\beta\end{pmatrix}=\begin{pmatrix}0,1,3,5,2,4,6,7\\0,1,1,0,1,0,1,0\end{pmatrix}$\parbox[c][40pt][c]{0pt}{}\tabularnewline
				\hline 
				Initialpopulation & $\beta_i=$ Zufallszahl aus $\{0,1\} \; \forall i \in \mathcal{J}$\tabularnewline
				\hline 
				Rekombination & One Point Crossover\tabularnewline
				\hline 
				Mutation & $\beta_i=\neg \beta_i$ mit $P_{mutate}$\tabularnewline
				\hline 
				Fitness & Deckungsbeitrag von kodiertem Plan\tabularnewline
				\hline 
			\end{tabular}
		\end{center}
	\end{small}
\end{frame}

%%%%%%%%%%%%%%%%%%%%%%%%%%%%%%%%%%%%%%%%%%%%%%%%%%%%%%%%%%%%%%%%%%%%%%%%%%%%%%%%%%%%%%%%%%%%%%%%%%%%%%%%%%%%%%%%%%%%%%%%

\begin{frame}[noframenumbering]
	\frametitle{Entscheidung über ZK in SGS einbauen}
	\begin{itemize}
		\item Bei Einplanung von AG $j$ Zeitfenster zwischen
		\begin{itemize}
			\item Reihenfolgezulässigkeit $\underline{t}$ und
			\item Ressourcenzulässigkeit $\overline{t}$
		\end{itemize}
		bestimmen und $ST_j \in \{ \underline{t}, \ldots, \overline{t} \}$ wählen, sodass ZK-Grenzen $\overline{z}_r$ eingehalten werden\\[8mm]
		\item Zwei Varianten
		\begin{enumerate}
			\item Auswahl durch GA 
			\item Teilplanvervollständigung mit SGS ohne ZK
		\end{enumerate}
	\end{itemize}
\end{frame}

\begin{frame}[noframenumbering]
	\frametitle{Ablauf Teilplanvervollständigung}
	Bei Einplanung von AG $j$:
	\vspace*{2mm}
	\begin{itemize}
		\item Für alle Perioden $t$ zwischen Reihenfolgezulässigkeit und Ressourcenzulässigkeit von AG $j$:
		\begin{itemize}
			\item Vervollständigung des Teilplans mit $ST_j=t$ und SSGS ohne Zusatzkapazität
			\item Bestimmung des Deckungsbeitrags\\[3mm]
		\end{itemize}
		\item Setze Startzeit von AG $j$ auf Periode, bei der vervollständigter Plan den Deckungsbeitrag maximiert
	\end{itemize}
\end{frame}

%%%%%%%%%%%%%%%%%%%%%%%%%%%%%%%%%%%%%%%%%%%%%%%%%%%%%%%%%%%%%%%%%%%%%%%%%%%%%%%%%%%%%%%%%%%%%%%%

%\section{Problembibliothek}
%\begin{frame}[noframenumbering]
%\frametitle{Gliederung}
%\tableofcontents[current] %, hidesubsections]
%\end{frame}

\begin{frame}[noframenumbering]
\frametitle{Parameter der Erlösfunktion}
\begin{itemize}
\item Problem: Bestimmung von $C^{\mbox{max}}$, $T^{\mbox{min}}$, $T^{\mbox{max}}$ erfordert Lösung eines RCPSP\\[5mm]
\item[$\rightarrow$]  Abschätzung mittels unterer bzw. oberer Schranken\\[2mm]
\begin{itemize}
\item $T^{\mbox{min}} \geq \mbox{max}\{EFT_{J},\mbox{max}_{r}\{\lceil\frac{\sum_{j}d_{j}k_{jr}}{K_{r}+\overline{z}_{r}}\rceil\}\}$\\[1mm]
\item $T^{\mbox{max}} \leq$ Dauer von SSGS-Ablaufplan mit $z_{rt}=0$\\[4mm]
\item $C^{\mbox{max}} \leq $ Kosten des $\mathcal{ESS}$
\end{itemize}
\end{itemize}
\end{frame}

\begin{frame}[noframenumbering]
\frametitle{Basis für Testinstanzen}
\begin{itemize}
\item Project Scheduling Problem Library (PSPLIB)
	\begin{itemize}
		\item RCPSP-Testinstanzen mit 30, 60, 90 und 120 AG
		\item Instanzgenerator PROGEN
		\item Beste bekannte Lösungen (Optimal für j30) für $T^{\mbox{max}}$\\[4mm]
	\end{itemize}

\item Erweiterung um $\kappa_r, \overline{z}_r$ und $u_t$\\[4mm]

\item Entferne Instanzen mit $T^{\mbox{min}} = T^{\mbox{max}}$
	\begin{itemize}
	\item $T^{\mbox{min}}$ kürzeste Projektdauer bei maximal möglicher ZK
	\item $T^{\mbox{max}}$ kürzeste Projektdauer bei keiner ZK
	\end{itemize}
\item Lösung von zwei RCPSPs je Testinstanz notwendig $\implies$ nur für $\lesssim 30$ AG praktikabel
\end{itemize}

\end{frame}

\begin{frame}[noframenumbering]
\frametitle{Berechnung optimaler Lösungen\\als Referenzwerte}
\begin{itemize}
\item Aktuell nur für 30 Arbeitsgänge\\[4mm]
\item GAMS-Implementierung von
	\begin{itemize}
	\item RCPSP-OC-Modell sowie
	\item RCPSP-Modell zur Berechnung von $T^{\mbox{min}}$\\[4mm]
	\end{itemize}
\item Für RCPSP-OC erweiterte j30-Testinstanzen in GDX-Format umgewandelt
\item Exakte Lösung per GUROBI parallel auf RRZN-Cluster
\item Problemspezifischer Branch\&Bound-Algorithmus
\end{itemize}
\end{frame}

\begin{frame}[noframenumbering]
\frametitle{Umsetzung der Genetischen Algorithmen}
\begin{itemize}
\item Implementierung in Delphi\\[4mm]
\item Großer Flaschenhals in $(\lambda)$-Repräsentation laut Profiling: Fitnessberechnung 
\begin{itemize}\item[$\rightarrow$] in Threads parallelisiert\\[4mm]\end{itemize}
\item Heuristik-Ausführung auf Cluster
\begin{itemize}
	\item Hoher Zeitbedarf für Gesamtevaluation:
		\begin{itemize}
		\item 12 Heuristiken
		\item 3*480 und 1*600 Projekte
		\item 1s, 2s, 10s, 15s Zeitlimit (für j30, j60, u.s.w.)
		\item[$\implies$] = $12 \cdot (480\cdot (1s + 2s + 10s) + 600\cdot15s ) = 50{,}8$ Stunden
		\end{itemize}
	\item Vergleichbarkeit
	\item Parallelisierte Berechnung gesamter Problembibliothek
\end{itemize}
\end{itemize}
\end{frame}

%%%%%%%%%%%%%%%%%%%%%%%%%%%%%%%%%%%%%%%%%%%%%%%%%%%%%%%%%%%%%%%%%%%%%%%%%%%%%%%%%%%%%%%%%%%%%%%%%%%%%%%%%

\begin{frame}[noframenumbering]
	\frametitle{Numerische Ergebnisse (Detailliert)}
	\begin{footnotesize}
		\textbf{Parameter:} Populationsgröße $=80$, $P_{mutate}=5\%$\\
		
		\begin{center}
			\begin{tabular}{ccc}
				\multicolumn{3}{c}{\textbf{Testinstanzen}}\\
				Datensatz & \#Projekte & Zeitlimit\\
				j30 & 199 & $1s$\\
				j120 & 586 & $15s$
			\end{tabular}
		\end{center}
		
		\begin{center}	
			\tabcolsep=0.16cm
			\begin{tabular}{|c|rrrc|rrrc|}
				\hline
				& \multicolumn{4}{c|}{j30} & \multicolumn{4}{c|}{j120}\\
				& $\varnothing$ Gap & $\overline{\mbox{Gap}}$ & Opt & $\varnothing$ Rang & $\varnothing$ Gap & $\overline{\mbox{Gap}}$ & BBL & $\varnothing$ Rang \\[3pt]
				\hline
				$(\lambda|z_{rt})$&1.41\%&\textbf{8.83\%}&46.73\%&1.98&1.91\%&12.38\%&21.20\%&3.63\\
				\hline
				$(\lambda|z_r)$&3.01\%&45.00\%&28.64\%&2.60&\textbf{0.95\%}&\textbf{8.30\%}&\textbf{52.14\%}&\textbf{2.19}\\
				\hline
				$(\lambda|\beta)$&2.80\%&45.00\%&29.65\%&2.37&1.61\%&8.31\%&25.98\%&2.97\\
				best & \multicolumn{4}{c|}{AG, Gem. OPC, oberes SGS} & \multicolumn{4}{c|}{Liste, Gem. OPC, unteres SGS}\\
				\hline
				$(\lambda|\beta)$&5.27\%&45.00\%&21.11\%&4.39&11.76\%&28.25\%&8.55\%&9.56\\
				worst & \multicolumn{4}{c|}{Liste, Getr. OPCs, oberes SGS} & \multicolumn{4}{c|}{Liste, Getr. OPCs, oberes SGS}\\
				\hline
				$(\lambda|\tau)$&2.45\%&20.23\%&41.21\%&2.98&4.48\%&14.17\%&2.56\%&6.08\\
				\hline
				$(\lambda)$& \textbf{1.20\%}&14.29\%&\textbf{52.76\%}&\textbf{1.78}&5.04\%&19.18\%&27.52\%&5.40\\
				\hline
			\end{tabular}
		\end{center}
		
		Erlösfunktion setzt minimal erreichbaren Deckungsbeitrag auf 0 $\implies$ Rationalskaliert
		
	\end{footnotesize}	
	
\end{frame}

\begin{frame}[noframenumbering]
\frametitle{Konvergenzverhalten}
\includegraphics<1>[page=1, scale=0.69]{images/Convergence3011_7.pdf}
\end{frame}

\begin{frame}[noframenumbering]
	\frametitle{Allgemeines serielles SGS}
	\begin{itemize}
		\item Planerzeugung in $J$ Schritten
		\begin{itemize}
			\item Bestimmung von einplanbaren AG\\$\mathcal{D}=\{ j \in \mathcal{J} | (\forall i \in \mathcal{P}_j \colon ST_i \neq \perp) \land ST_j = \perp\}$\\[2mm]
			\item $j \in \mathcal{D}$ wählen
			\begin{itemize}
				\item Prioritätsregel
				\item Prioritätswerte
				\item Sampling\\[2mm]
			\end{itemize}
			\item $j$ frühstmöglich einplanen
			\begin{itemize}
				\item Alle Vorgänger beendet
				\item Genügend Ressourcenkapazität während Durchführung
			\end{itemize}
		\end{itemize}
		\item Spezialfall: Serielles SGS mit $\lambda$ als Eingabe
	\end{itemize}
\end{frame}

\begin{frame}[noframenumbering]
	\frametitle{Beispiele für verwendete\\Auswahlverfahren}
	\begin{itemize}
		\item Prioritätsregel $LFT, SPT, MIS, MTS, \ldots$\\[4mm]
		\item Prioritätswerte $v=(0.1, 0.3, \ldots)$\\[4mm]
		\item Random sampling $p_j=\frac{1}{|\mathcal{D}|}$\\[4mm]
		\item Sonderfall: Aktivitätsliste $\lambda=(0, 1,2,\ldots)$
	\end{itemize}
\end{frame}

\begin{frame}[noframenumbering]
	\frametitle{Regret Based Biased Random Sampling}
	
	\begin{itemize}
		\item Biased random sampling $ p_j = \frac{v_j}{\sum_i v_i} $ \\[5mm]
		\item Regret value \[r_j = v_j - \min_i v_i\]
		\item Adjusted regret value \[ r_j^\prime = (r_j + \epsilon)^\alpha \]
		\item Häufig $\epsilon = \alpha = 1$
		\item Auswahlwahrscheinlichkeit \[p_j = \frac{r_j^\prime}{\sum_i r_i^\prime} = \frac{v_j - \min_i v_i + 1}{n + \sum_i v_i - \min_k v_k}\]
	\end{itemize}
\end{frame}

\begin{frame}[noframenumbering]
	\frametitle{Korrespondenz $\lambda \leftrightarrow $ Ablaufplan}
	\begin{itemize}
		\item $ST_j(\lambda)$: AL zu Plan über serielles SGS\\[6mm]
		\item $\lambda(ST_j)$: Plan zu AL über Vorschrift
		\begin{itemize}
			\item $ST_i < ST_j \implies \lambda_i < \lambda_j$
			\item $ST_i=ST_j \land i < j \implies \lambda_i < \lambda_j$\\[6mm]
		\end{itemize}
		\item Beachte: $\lambda(ST_j(\lambda)) \neq \lambda$ möglich
	\end{itemize}
\end{frame}

\begin{frame}[t,noframenumbering]
	\frametitle{Beispiel für Startpopulationserzeugung}
	\only<1>{
		\begin{center}
			\includegraphics[scale=1.0]{images/ProjektstrukturStartpopulation.pdf}\\
			\begin{small}$\mathcal{R}=\{1\}, K_1=4$\end{small}\\[5mm]
			Probleminstanz
		\end{center}
	}
	\only<2>{
		\begin{center}
			Abgeleitete Größen\\[5mm]
			\begin{tabular}{|c||c|c|c|c|c|c|}
				\hline 
				$j$ & 0 & 1 & 2 & 3 & 4 & 5\tabularnewline
				\hline
				\hline 
				$EFT_{j}$ & 0 & 1 & 3 & 5 & 3 & 5\tabularnewline
				\hline 
				$LFT_{j}$ & 0 & 3 & 3 & 5 & 5 & 5\tabularnewline
				\hline 
				$v_{j}$ & 0 & -3 & -3 & -5 & -5 & -5\tabularnewline
				\hline 
				$r_{j}$ & 5 & 2 & 2 & 0 & 0 & 0\tabularnewline
				\hline 
				$r_{j}^{\prime}$ & 6 & 3 & 3 & 1 & 1 & 1\tabularnewline
				\hline
			\end{tabular}
			\\[8mm]$v_j = -LFT_j$ und $\epsilon = \alpha = 1$
		\end{center}
	}
	\only<3-9>{
		\begin{center}
			\begin{small}
				\only<3>{
					\includegraphics[scale=0.75,page=1]{images/RBBRS-Beispiel.pdf}\\
					\textbf{Schritt 1:} $ST_j=(\perp, \perp, \perp, \perp, \perp, \perp)$\\$\mathcal{D}=\{0\}, p_1=1 \implies ST_0=0$
				}
				
				\only<4>{
					\includegraphics[scale=0.75,page=1]{images/RBBRS-Beispiel.pdf}\\
					\textbf{Schritt 2:} $ST_j=(0, \perp, \perp, \perp, \perp, \perp)$\\$\mathcal{D}=\{1, 2\}, p_1=p_2=\frac{3}{6}=\frac{1}{2} \implies ST_2=0$
				}
				
				\only<5>{
					\includegraphics[scale=0.75,page=2]{images/RBBRS-Beispiel.pdf}\\
					\textbf{Schritt 3:} $ST_j=(0, \perp, 0, \perp, \perp, \perp)$\\$\mathcal{D}=\{1\}, p_1=1 \implies ST_1=3$
				}
				
				\only<6>{
					\includegraphics[scale=0.75,page=3]{images/RBBRS-Beispiel.pdf}\\
					\textbf{Schritt 4:} $ST_j=(0, 3, 0, \perp, \perp, \perp)$\\$\mathcal{D}=\{3, 4\}, p_3=p_4=\frac{1}{2} \implies ST_4=4$
				}
				
				\only<7>{
					\includegraphics[scale=0.75,page=4]{images/RBBRS-Beispiel.pdf}\\
					\textbf{Schritt 5:} $ST_j=(0, 3, 0, \perp, 4, \perp)$\\$\mathcal{D}=\{3\}, p_3=1 \implies ST_3=4$
				}
				
				\only<8>{
					\includegraphics[scale=0.75,page=5]{images/RBBRS-Beispiel.pdf}\\
					\textbf{Schritt 6:} $ST_j=(0, 3, 0, 4, 4, \perp)$\\$\mathcal{D}=\{5\}, p_5=1 \implies ST_5=6$
				}
				
				\only<9>{
					\includegraphics[scale=0.75,page=5]{images/RBBRS-Beispiel.pdf}\\
					\textbf{Ergebnis:} $ST_j=(0, 3, 0, 4, 4, 6)$ \\ $\implies \lambda = (0, 2, 1, 3, 4, 5)$
				}
			\end{small}
		\end{center}
	}
\end{frame}

\begin{frame}[noframenumbering]
	\frametitle{Paarbildung}
	\includegraphics[scale=1.0]{images/Paarbildung.pdf}\\[3mm]
	\begin{itemize}
		\item Wähle aus besten 50\% sukzessive 1. Elternteil
		\item Wähle aus schlechten 50\% zufällig 2. Elternteil
		\item Jeweils aus Elternpopulation ``streichen''
	\end{itemize}
\end{frame}

\begin{frame}[noframenumbering]
	\frametitle{Peak Crossover}
	\begin{itemize}
		\item \textbf{Grundidee:} Tochter (Sohn) übernimmt Peaks von Mutter (Vater) und restliche AG in Reihenfolge von Vater (Mutter)
		\item Auslastungsverhältnis in Periode $t$:
		\[\mbox{RUR}(t)=\frac{1}{|\mathcal{R}|} \sum_{r \in \mathcal{R}} \frac{\sum_{j \in \mathcal{S}(t)} k_{jr}}{K_r}\]
		\item Intervall $I$ mit hoher Auslastung $\mbox{RUR}(t)\geq \delta \; \forall t \in I$
		\item $\delta = \mbox{rand} [lthreshold, uthreshold]$
		\item Peaks $P_1,\ldots,P_q$
		\begin{itemize}
			\item $P_k$ enthält durchgeführte Jobs in $I_k$
			\item Teilsequenzen aus $\lambda$
			\item Nicht zwangsläufig zusammenhängend
		\end{itemize}
	\end{itemize}
\end{frame}

\begin{frame}[t,noframenumbering]
	\frametitle{Beispiel für Peak Crossover}
	\begin{center}
		\includegraphics<1>[scale=0.7,page=1]{images/PeakCrossoverExample.pdf}
		\includegraphics<2>[scale=0.7,page=2]{images/PeakCrossoverExample.pdf}
		\includegraphics<3>[scale=0.7,page=3]{images/PeakCrossoverExample.pdf}
		\includegraphics<4>[scale=0.7,page=4]{images/PeakCrossoverExample.pdf}\\[2mm]
		\only<1>{
			$\lambda^m=(0, 1, 3, 2, 4, 5, 6, 7, 8, 9, 10)$\\
			$I_1=\{3, 4, 5\}, I_2=\{7, 8, 9\}$\\
			$P_1=(3, 2, 4, 5), P_2=(6, 7, 8, 9)$
		}
		\only<2>{
			$\lambda^f=(0, 3, 4, 6, 7, 1, 9, 2, 5, 8, 10)$
			$I_1=\{1, 2\}, I_2=\{3, 4, 5\}$\\
			$P_1=(3, 4), P_2=(6, 7, 1, 9)$
		}
		\only<3>{
			$\lambda^d=(0, 3, 2, 4, 5, 1, 6, 7, 8, 9, 10)$
		}
		\only<4>{
			$\lambda^s=(0, 3, 4, 6, 7, 1, 9, 2, 5, 8, 10)$
		}
	\end{center}
\end{frame}



\begin{frame}[noframenumbering]
	\frametitle{Vorwärts Rückwärts Verbesserung (FBI)}
	\begin{itemize}
		\item Weit verbreitet in führenden RCPSP-Heuristiken 
		\item Auch als ``Doppeljustierung'' (DJ) bezeichnet\\[5mm]
		\item \textbf{1. Schritt:} Rückwärts von $J-1$ restliche AG so weit wie möglich nach rechts verschieben
		\item \textbf{2. Schritt:} Vorwärts von $0$ restliche AG so weit wie möglich nach links verschieben\\[5mm]
		\item Häufig höhere Auslastung und kürzere Projektdauer
	\end{itemize}
\end{frame}

\begin{frame}[noframenumbering]
	\frametitle{Beispiel für FBI}
	\only<1>{
		\begin{center}
			\includegraphics[scale=1.0]{images/ProjektstrukturStartpopulation.pdf}\\
			\begin{small}$\mathcal{R}=\{1\}, K_1=4$\end{small}\\[5mm]
			Probleminstanz
		\end{center}
	}
	\only<2-4>{
		\begin{center}
			\includegraphics<2>[page=1,scale=0.8]{images/FBIBeispiel.pdf}
			\includegraphics<3>[page=2,scale=0.8]{images/FBIBeispiel.pdf}
			\includegraphics<4>[page=3,scale=0.8]{images/FBIBeispiel.pdf}\\
			\only<2>{Ausgangssituation}
			\only<3>{Rechtsjustierung}
			\only<4>{Linksjustierung}
		\end{center}
	}
\end{frame}

\begin{frame}[noframenumbering]
	\frametitle{Umsetzung FBI mithilfe SGS}
	\begin{itemize}
		\item Präzedenz invertieren $\mathcal{P}_j^\prime = \mathcal{P}_j^{-1} = \{ i \in \mathcal{J} | j \in \mathcal{P}_i \}$
		\item AL drehen $\lambda_j^\prime = \lambda_{J-j}$
		\item[$\rightarrow$] Rechtsjustierter Plan $ST_j^\prime= ST_j(\mathcal{P}_j^\prime, \lambda_j^\prime)$\\[8mm]
		\item Plan spiegeln $ST_j^M = ST_0^\prime - (ST_j^\prime + d_j)$
		\item AL ableiten $\lambda_j^{\prime\prime} = \lambda(ST_j^M)$
		\item[$\rightarrow$] Linksjustierter Plan $ST_j^{\prime\prime} = S(\lambda_j^{\prime\prime})$
	\end{itemize}
\end{frame}

\begin{frame}[noframenumbering]
	\frametitle{Zweite Suchphase}
	\begin{itemize}
		\item Populationsgröße halbieren\\[4mm]
		\item Startpopulation:\\Nachbarschaft bester Lösung aus Suchphase 1\\[4mm]
		\item Erzeugung der Nachbarschaft:\\$\beta$-Biased Random Sampling
	\end{itemize}
\end{frame}

\begin{frame}[noframenumbering]
	\frametitle{$\beta$-Biased Random Sampling}
	\begin{itemize}
		\item $\beta=1-\frac{20}{J}$
		\item Auswahlregel für $j \in \mathcal{D}$ im seriellen SGS
		\item Ziehe Zufallszahl $p \in (0,1)$ gleichverteilt
		\item Falls $p < \beta$ wähle $k = \lambda_i \in \mathcal{D}$ sodass $i$ minimal
		\item Sonst wähle $j$ per biased random sampling mit $v_j=\lambda_j^{-1}$ aus $\mathcal{D} \setminus \{ k\}$
		\item Überstunden analog zu RBBRSM einbringen
	\end{itemize}
\end{frame}

